% \RequirePackage[ngerman=ngerman-x-latest]{hyphsubst}
\documentclass[english, BCOR=6mm, twoside=true, open=right]{tudscrreprt}
\usepackage[utf8]{inputenc}
\usepackage[english]{babel}
\usepackage{selinput}
\usepackage[T1]{fontenc}
\usepackage{isodate}
\usepackage{graphicx}
\usepackage{pst-all}
\usepackage{mathtools}
\usepackage{latexsym}
\usepackage{amssymb}
\usepackage{threeparttable}
\usepackage{glossaries}

\makeglossaries
\newglossaryentry{computer}
{
  name=computer,
  description={is a programmable machine that receives input,
               stores and manipulates data, and provides
               output in a useful format}
}
\newglossaryentry{new computer}
{
  name=new computer,
  description={is a programmable machine that receives input,
               stores and manipulates data, and provides
               output in a useful format}
}
\newglossaryentry{old computer}
{
  name=old computer,
  description={is a programmable machine that receives input,
               stores and manipulates data, and provides
               output in a useful format}
}


\begin{document}

\faculty{Department of Computer Science}
% \department{Fachrichtung Strafrecht}
\institute{Institute for Systems Architecture}
\chair{Chair of Computer Networks}
\date{18.02.2015}
\title{Graphical Discussion System}
\subject{master}
\graduation[M.Sc.]{Master of Science}
\author{%
  Kaijun Chen
  \matriculationnumber{3942792}
  \dateofbirth{18.09.1990}
  \placeofbirth{China}
}
\matriculationyear{2013}
\supervisor{Tenshi Hara \and Iris Braun}
\professor{Prof. Dr. rer. nat. habil. Dr. h. c. Alexander Schill}

\maketitle

\confirmation

% empty page
% \clearpage
% \thispagestyle{empty}
% \hfill
% \clearpage

% Multiple Abstract
% \TUDoption{abstract}{section,multiple}
% \begin{abstract}[pagestyle=empty.tudheadings]
% \input{Content/01_Abstract_De}
% \nextabstract[english]
% \input{Content/01_Abstract_En}
% \end{abstract}

\TUDoption{abstract}{section}
\begin{abstract}[pagestyle=empty.tudheadings]
\input{Content/01_Abstract}
\end{abstract}

\tableofcontents

% \chapter{Einleitung}
% What's the situation now?

Pain?

% \input{Content/12_Vorgehen}
% \input{Content/13_Notation}\label{Notation}
% % \input{Content/13_Notation}
%
% \chapter{Near-Optimale MDS-Codes}\label{nearMDS}
% \input{Content/201_Metric}
% \section{Regenerating Codes (RC)}
% \input{Content/211_ParameterVonRC}\label{Parameter}
% \input{Content/212_MBRundMSR}\label{MBRuMSR}
% \input{Content/213_FunktionaleRegenerating}\label{FunkRe}
% \input{Content/214_ExakteRegenerating}\label{ExaRe}
% \input{Content/215_VergleichFRCundERC}
% \section{Locally Repairable Codes (LRC)}
% \input{Content/22_LRC}
% \section{Zusammenhang zwischen RC und LRC}
% \input{Content/23_Zusammenhang}
%
% \chapter{Implementierung eines Algorithmus für exaktes Regenerating}\label{Implementierung}
% \input{Content/31_Implementierung}
% \input{Content/32_ImplementierungMBR}\label{ImplementierungMBR}
% \input{Content/321_NachrichtenmatrixMBR}\label{NachrichtMBR}
% \input{Content/322_NichtSysGeneratormatrixMBR}\label{NSGeMaMBR}
% \input{Content/323_SysGeneratormatrixMBR}\label{SGeMaMBR}
% \input{Content/324_KodierungMBR}\label{KodierungMBR}
% \input{Content/325_RegeneratingMBR}
% \input{Content/326_DekodierungMBR}
% \input{Content/33_ImplementierungMSR}\label{ImplementierungMSR}
% \input{Content/331_NachrichtenmatrixMSR}
% \input{Content/332_NichtSysGeneratormatrixMSR}\label{NSGeMaMSR}
% \input{Content/333_SysGeneratormatrixMSR}
% \input{Content/334_KodierungMSR}\label{KodierungMSR}
% \input{Content/335_RegeneratingMSR}
% \input{Content/336_DekodierungMSR}
%
%
% \chapter{Experimentelle Untersuchung und Auswertung}\label{Untersuchung}
% \input{Content/41_Dimensionierung}
% \section{Auswertung an Speicher, Reparatur-Bandbreite und Disk-I/O}
% \input{Content/421_Speicher}
% \input{Content/422_Reparatur-Bandbreite}
% \input{Content/423_Disk-IO}
% \input{Content/43_Vergleich}
%
% \chapter{Schluss}\label{Schluss}
% \input{Content/51_ProblemBesonderheit}
% \input{Content/52_WeiterfuehrendeBetrachtungen}
%
% \newpage
% \addcontentsline{toc}{chapter}{Abbildungsverzeichnis}
% \listoffigures
%
% \newpage
% \addcontentsline{toc}{chapter}{Tabellenverzeichnis}
% \listoftables
%
% \newpage
% \addcontentsline{toc}{chapter}{Glossar}
% \printglossary
%
% \newpage
% \addcontentsline{toc}{chapter}{Literaturverzeichnis}
% \bibliographystyle{ieeetr}
% \bibliography{references}

\end{document}
