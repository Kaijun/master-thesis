Usability testing refers to evaluating a product or service by testing it with representative users. In principle, during a usability test, participants  will evaluate the system with quantitive metrics. 

The goal of usability testing is to collect the quantitive data, analyse the result and issue the usability problems with tested system. 

\subsection{System Usability Scale}

The System Usability Scale (SUS) offers a "quick and dirty", but relative reliable approach for measuring the usability\cite{brooke1996sus}.  It contains a 10 item questionnaire with five rating options for participants; from \textit{strongly agree} to \textit{strongly disagree}.


In order to calculate the SUS score, score contributions from each item should be calculated once separately at first. Score contribution will range from 0 to 4. For items with odd number, the score contribution should minus 1. For item with even number, the contribution is 5 minus the score. The sum of all scores is multiplied by 2.5 to obtain the overall value of SUS, which has a range of 0 to 100.

\begin{equation}
\label{formular:SUS}
SUS_{sum} = 2.5 \times \left (  \sum_{ i=1}^{5} \left ( a_{2i-1} - 1 \right ) + \sum_{ i=1}^{5} \left ( 5 - a_{2i} \right ) \right ) 
\end{equation}


\begin{table}[!htbp]
\centering
\begin{tabularx}{\textwidth}{@{}lXXXXXl@{}}
\toprule
Item(No.)   & A  & B  & C  & D  & E          & Average Score        \\ \midrule
1               & 3  & 4  & 3  & 4  & 4          & 3.6                     \\
2               & 2  & 1  & 1  & 1  & 2          & 1.4                     \\
3               & 5  & 4  & 4  & 5  & 4          & 4.4                     \\
4               & 1  & 2  & 1  & 2  & 2          & 1.6                     \\
5               & 4  & 4  & 3  & 4  & 5          & 4                     \\
6               & 4  & 5  & 4  & 3  & 4          & 4                     \\
7               & 5  & 4  & 4  & 5  & 5          & 4.6                     \\
8               & 3  & 3  & 2  & 1  & 3          & 2.4                     \\
9               & 5  & 4  & 3  & 4  & 5          & 4.2                     \\
10              & 2  & 3  & 2  & 1  & 2          & 2                     \\ \bottomrule        
\end{tabularx}
\caption{Score of SUS table}
\label{table:score-sus}
\end{table}

For the SUS testing of Graphicuss system, 5 participants are involved in the interview with SUS questionnaire, which is listed in appendix \ref{appendix:sus}. Each participant gives his own score contribution for each item, and the average score of each item is calculated. Table \ref{table:score-sus} shows the result.

According to the formula \ref{formular:SUS}, the final sum SUS score of the system is \textbf{73.5}. An article represents the mapping of adjective ratings to SUS score\cite{bangor2009determining}. And a \textbf{73.5} SUS score achieves the rating in a range of \textit{Good} to \textit{Excellent} when it is expressed by adjective ratings. 

The result reveals that the Graphicuss system achieves a relative high score in the general usability test. In general, users are able to learn to use this system very quickly. Without significant help, they can operate the system smoothly and unproblematically. 

% \cite{bangor2009determining} \cite{gackenheimer2015core}\cite{barron1998minimum}\cite{grunwald2005advances}\cite{ferraiolo2000scalable}\cite{fette2011websocket}\cite{geary2012core}\cite{richardson2008restful}\cite{pautasso2008restful}\cite{pimentel2012communicating}



\subsection{Interview based Usability Test}

Interviews with college students have been conducted in order to evaluate the system's usability as well as the fulfilment of requirements. 

As a research performed by Nielsen, 

\begin{itemize}
  \item  Interview type: Discussion and questions to be answered using a 5-point Likert scale with additional space for comments and feedback.
  \item  Number of questions: 8
  \item  Duration of each interview: 20-30 minutes
  \item  Time period of the conduction: 27th June 2016. The question sheet is attached to this thesis in Appendix B.
  \item  Interview conduction: 
  \begin{enumerate}
    \item Several courses are created at the very beginning before the interview is performed. 
    \item After the interview is started, interviewees are requested to sign up with their own accounts and search the certain course by the course code.
    \item Afterwards, they start questioning or answering within the certain course at the same time.
    \item Interviewees are also demanded to use the major functionalities of the system, especially the drawing tool and quote functionality.
    \item At last, the remaining 8 questions have been answered by the interviewees, followed by a final discussion of the results.
  \end{enumerate}
\end{itemize}

The overall results of the interviews are presented in table \ref{}, which contains the score rated by each interviewee and the average score of each question. 

\subsubsection{Analyse General Functionalities}
\subsubsection{Analyse of Drawing Tool }
\subsubsection{Analyse of Real-Time Functionality}