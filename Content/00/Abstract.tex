A discussion of the teaching content or the educational material is always essencial for both tutors and students in the teaching activities. In traditional way, a discussion can only be performed normally after courses also requires the absence of the students as well as the tutors.

The traditional approach of discussing shows its limitations. Inefficiency in knowledge acquisition: not all the students have the same question and the tutor is able to provide explanation for only one question at same time; time-consumption: ; low interactivity:

Thus, a discuss system with intense interactivity as well as in crowdsourcing way is highly needed. To achieve high interactivity, a discuss system with graphical tool and real-time data communication is proposed. Students are able to contribute their questions and answers to get to the bottom of his deficiencies of teaching content and the educational material. And students who has the same questions can instantly acquire the best solution  which is recommended and approved by the community.

In order to validate and evaluate the concepts of this approach, an implementation of the proposed solution is developed on top of modern web technologies. Moreover, a usability questionnaire survey is proposed and delivered for a quantized evaluation of the client application. The performance of this application is also evaluated at the same time through the created simulation scenarios.
