Before starting with the concept the graphical discuss system, it's necessary to analyze requirements and objectives behind the origin motivation in the first place. It should be defined at first, what kind of functionalities should be achieved and how the system behaves.

\subsection{Basic Functionalities}

As a graphical discuss system for the educational purpose, the system should contain basic functionalities on the prototype  of a forum which could be organized by classes. So class management, question management and answer management are the three essential parameters to be designed at the start.

\subsubsection{Course Management}

Each question should have a certain domain of its content, so the questions are organized by classes initially. The features of course management should be:

\begin{enumerate}
\item
\textbf{Create Course}: The user who is identified as a tutor is able to create courses and maintain the courses he created. While creating the course, the tutor can define the name of the course and upload an image as a background of the course for better recognition. In addition, concrete description of the course could also be added to the description area.

\begin{figure}[!htbp]
  \centering
    \includegraphics[width=0.8\textwidth]{Figures/mockup/add-new-course.pdf}
  \caption{Submit a new course}
\end{figure}

% Mockups create, upload, description

\item
\textbf{Search Course}: After a course is created, a corresponding unique identifier code for the course will be generated at the same time. The students are able to find the course through the identifier code.

\begin{figure}[!htbp]
  \centering
    \includegraphics[width=0.8\textwidth]{Figures/mockup/Search-Course.pdf}
  \caption{Search course with code}
\end{figure}
% Mockups code, search

\item
\textbf{Favor Course}: If a student is interested in a certain course, he is capable to add the course to his favorites list so that it's easy to find and access the course he liked later.

\begin{figure}[!htbp]
  \centering
    \includegraphics[width=0.8\textwidth]{Figures/mockup/Favour-Course.pdf}
  \caption{Favor course}
\end{figure}
% Mockups fav button, fav list.

\end{enumerate}

\subsubsection{Question Management}

\begin{enumerate}
\item
\textbf{Submit/Edit/Withdraw Question}: The student who is confused with the teaching content can submit his own question with detailed description in a certain course. The user is also permitted to edit the question if he wants to add more precise informations or modify the unclarity he made to the question. Withdrawing of his own question is also possible, but only when there're no contributes made to the question.

\begin{figure}[!htbp]
  \centering
    \includegraphics[width=0.8\textwidth]{Figures/mockup/New-question.pdf}
  \caption{Submit a new question; withdraw or modify own question}
\end{figure}
% Mockups submit/Edit, withdraw

\item
\textbf{Upvote/Downvote Question}: An assessment of a question is decisive for building a better community with high-quality contents. So the user is able to upvote or downvote of a question and determines if the question is helpful for other members in the community or not.

\begin{figure}[!htbp]
  \centering
    \includegraphics[width=0.8\textwidth]{Figures/mockup/question-vote.pdf}
  \caption{Upvote/Downvote a question or answer}
\end{figure}
% Mockups Upvote/downvote.

\item
\textbf{Favor Question}: If the student considers the question as a helpful and useful content and want to review this question in the future, he can favor the question and locate it in a certain list.

% \begin{figure}[!htbp]
%   \caption{placeholder}
%   \centering
%     \includegraphics[width=0.8\textwidth]{Figures/placeholder.png}
%   \label{fig:placeholder}
% \end{figure}
% Mockups Fav, fav list

\item
\textbf{Accept Answer}: The owner of the question has the right to accept the most useful answer in his opinion, which will be shown up at the top of the answer list.

% \begin{figure}[!htbp]
%   \caption{placeholder}
%   \centering
%     \includegraphics[width=0.8\textwidth]{Figures/placeholder.png}
%   \label{fig:placeholder}
% \end{figure}
% Mockups Accept, Top.

\end{enumerate}

\subsubsection{Answer Management}

\begin{enumerate}
\item
\textbf{Submit/Modify/Remove Answer}: User who has experience with the question can submit his answer to the question. After the submission, the modification or removal of the user's own question is possible.

% \begin{figure}[!htbp]
%   \centering
%     \includegraphics[width=0.8\textwidth]{Figures/mockup/New-Answer-modify.pdf}
%   \caption{Submit a new answer}
% \end{figure}
% Mockups submit, withdraw

\item
\textbf{Upvote/Downvote Answer}: As mentioned above in subsection of question functionality, a similar idea of assessment should also be applied to answers. Answer with the highest vote will be listed at first.

% \begin{figure}[!htbp]
%   \caption{placeholder}
%   \centering
%     \includegraphics[width=0.8\textwidth]{Figures/placeholder.png}
%   \label{fig:placeholder}
% \end{figure}
% Mockups Upvote/downvote, arrange of answer.

\item
\textbf{Quote Answer}: Answers are able to be quoted so that the user can supplement information on the top of the original post or point out the deficiency of the contribute.

\begin{figure}[!htbp]
  \centering
    \includegraphics[width=0.8\textwidth]{Figures/mockup/quote.pdf}
  \caption{Submit/Quote an answer}
\end{figure}
% Mockups Fav, fav list

\end{enumerate}


\subsection{High Interactivity}
Building with the basic functionalities is far not enough. To fit the system for educational purpose and improve the interactivity for arousing enthusiasm of students, a drawing tool and real-time functionality should be integrated into the system.

\subsubsection{Drawing Tool}
Normally, some of the thoughts can't be simply expressed by textual description, so a drawing tool should be designed to enable the user to compose not only text but also different components such like rectangle, circle, line and so on, which helps the user to express his question more precisely.
The ideal drawing tool should have following features:

\begin{enumerate}
\item
\textbf{Drawing Diverse Components}: Not only text but also diverse components could be drawn while posting a contribution. The styling of a component such as size tuning, color changing is also the essential, which will help emphasize the important part the user expressing.

\item
\textbf{Drawing History}: During drawing, the user might make mistakes or change mind after placing a component or text. So a history list of drawing actions bundled with undo and redo functionalities will dramatically improve the usability of drawing process.

\begin{figure}[!htbp]
  \centering
    \includegraphics[width=0.8\textwidth]{Figures/mockup/editor.pdf}
  \caption{Drawing editor with drawing history}
\end{figure}
% Mockups history list, undo, redo


\end{enumerate}


\subsubsection{Real-time}
How to ease the approach of content acquisition and improve the interactivity for arousing enthusiasm of students, is also a key point while designing the discuss system. So two major real-time functionalities are featured as follows: 

% Auto Ordering of Questions / Answers
% Sidebar notifications!

\begin{enumerate}
\item
\textbf{Real-time Question List}: Without requesting the question list initiatively, all new questions posted by other users will be pushed to user automatically. The user doesn't have to concern himself with acquisition of the new content anymore.

\begin{figure}[!htbp]
  \centering
    \includegraphics[width=1\textwidth]{Figures/mockup/question-notify.pdf}
  \caption{Notify with new question automatically}
\end{figure}

\item
\textbf{Real-time Answer Ordering}: Without refreshing the page, the answers will be re-ordered as new vote action is triggered.

\begin{figure}[!htbp]
  \centering
    \includegraphics[width=1\textwidth]{Figures/mockup/votechange.pdf}
  \caption{Auto re-order answer if vote contributions changed}
\end{figure}
% Mockups Upvote/downvote, arrange of answer.


\end{enumerate}
