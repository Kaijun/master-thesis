Before the whole conception of the system, a general conceptual architecture of the system should be defined initially. In order to help understanding how the system works, the primary data flow between different domains will also be described.

\subsection{Architecture} \label{sec:concept-general-architecture}
According to the analysis result of Single-Page-App in chapter \ref{chapter:state-of-the-art}, and considering the demand on high interactivity and ubiquity as well as scalability in the graphical discuss system, leveraging SPA architecture will benefit a lot and accelerate the implementation of the system. 

In general, the entire system will be divided into two parts: namely client and server-side. Each side is basically full independent to the other and has its own responsibility. 

\begin{itemize}
\item
\textbf{Client}: The client is totally responsible for initial view rendering and view re-rendering as the view model changes. 
\item
\textbf{Server}: The server is in charge of core business logic, data processing, data persistence and also provides the client interfaces for data acquisition.
\end{itemize}

% General Architecture of System including Server(data persistence with db) + Client(has multiple views, one template engine, re-render with new data.)
\begin{figure}[!htbp]
  \centering
    \includegraphics[width=1\textwidth]{Figures/concept-general-architecture.pdf}
  \caption{General architecture in conception}
  \label{fig:general-architecture-concept}
\end{figure}

General architecture of the system is described in figure \ref{fig:general-architecture-concept}, the only bridge between the client and server-side is data transmission service. Complete separation of both sides will also accelerate the development flow in implementation phase. Once the protocol of data transmission services is fully confirmed and defined, development of each side is able to performed parallelly. Furthermore, technical choices on both sides are more flexible. Both sides are able to apply the technologies which fit them most without coupling to each other, the only thing they should obey is to follow the protocol of data transmission.

\subsection{Communication} \label{subsection:concept-general-communication}
As mentioned above in section \ref{sec:concept-general-architecture}, data communication is the only coupling factor in the general architecture. In this system, there exist two different type of protocols: standard HTTP using REST architecture and WebSocket with persistent connection. Each data transmission protocol has its own responsibility and usage scenario.

\begin{itemize}
\item
\textbf{HTTP with REST architecture}: Data which is requested initiatively is transferred over HTTP. The HTTP connection will be closed as soon as the data is successfully transferred.
\item
\textbf{WebSocket with persistent connection}: Reactive data with realtime need is transferred over WebSocket. After the persistent connection is established, client are able to receive the data at the first moment as the state of data is updated. 
\end{itemize}

% Pic broadcast to multiple clients with different events. engine to rerender
\begin{figure}[!htbp]
  \centering
    \includegraphics[width=0.8\textwidth]{Figures/concept-general-data-communication.pdf}
  \caption{General data communication}
  \label{fig:general-data-communication-concept}
\end{figure}

As figure \ref{fig:general-data-communication-concept} shows, in case data for view model is acquired from server despite transferred over HTTP or WebSocket, the views are re-rendered. Comparing with HTTP, the specialty of data acquisition over WebSocket is: a listener for specific resource with unified process of data processing and automatic view re-rendering will be created.