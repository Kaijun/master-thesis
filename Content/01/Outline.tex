This master thesis has the following structure.

\textbf{Chapter 2} gives general introductions of modern Web technologies. The new development process and architecture of Web application are discussed. Afterwards various graphics on the Web are presented and compared. Finally, alternatives of real-time communication technologies are listed.

% \textbf{Chapter 2} shows the latest Android APIs that can be used to access important network parameters. Some research findings about the system information on smartphones are demonstrated and how this information can be accessed. The next section is about some of my own measurements and finally problems occurring during measurements and accuracy of the results are discussed.

\textbf{Chapter 3} considers the general concept of the system. The first section deals with the requirements and mockups with expected functionalities. Thereafter the conception of general architecture is described. In addition, the concept of data model within the system is defined. Finally a feasible concept of serialized graphical data for storing is designed.

\textbf{Chapter 4} covers the implementation of both client and server application of the graphical discuss system. Firstly, general overview about the application structure is given. The storage structure and relation of the data model are shown. At last, the implementation of drawing tool which provides user interfaces for drawing is presented.

\textbf{Chapter 5} obtains the evaluation of system usability and graphical data model. The measurement methodology is introduced and test results are analyzed. 

\textbf{Chapter 6} is the epilogue with a summary of the thesis. The final section discusses future work or researches.